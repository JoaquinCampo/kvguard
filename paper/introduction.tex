% kvguard paper — Abstract and Introduction
% Standalone file; % kvguard paper — Abstract and Introduction
% Standalone file; % kvguard paper — Abstract and Introduction
% Standalone file; % kvguard paper — Abstract and Introduction
% Standalone file; \input{introduction.tex} from main paper
% Requires: No special packages

\begin{abstract}
KV-cache compression enables long-context inference under memory constraints, but existing compressors operate without feedback: they evict tokens irreversibly and never verify whether eviction degraded generation quality.
We show that compression induces \emph{catastrophic failures}---looping, non-termination, and reasoning corruption---that standard average-accuracy metrics hide.
We present \textsc{kvguard}, a closed-loop controller inspired by TCP congestion control that wraps any existing KV-cache compressor.
\textsc{kvguard} extracts a 42-dimensional per-token feature vector from the model's own logit distribution at near-zero cost, trains a hazard predictor to forecast catastrophe within the next 32 tokens (AUROC 0.945), and uses a hysteresis-based state machine to dynamically relax compression when danger is detected.
Evaluated on Qwen2.5-3B-Instruct with GSM8K across three compressors (StreamingLLM, SnapKV, ObservedAttention) at five compression ratios, \textsc{kvguard} reduces the Catastrophic Failure Rate by 43.2\% overall (63.9\% excluding immediate-onset cases) with only 6.9\% false positive rate.
The controller exceeds 50\% CFR reduction on all three compressors, validating its compressor-agnostic design.
Ablations confirm that each component---trained predictor (+42.6pp over random), hysteresis ($-$53pp false positives), and graduated response---contributes independently.
\end{abstract}


\section{Introduction}
\label{sec:introduction}

Autoregressive language models cache key-value (KV) pairs from every attention layer at every decoding step, producing memory requirements that grow linearly with sequence length.
For long-context inference---multi-step reasoning, document summarization, agentic workflows---this cache can dominate GPU memory, prompting a line of work on KV-cache compression: methods that evict, quantize, or merge cache entries to reduce the memory footprint~\cite{streamingllm2024,snapkv2024,h2o2023,pyramidkv2024}.
These methods report average accuracy or perplexity under compression and show that significant memory savings are possible with modest average-case degradation.

However, average metrics conceal a dangerous failure pattern.
At moderate-to-aggressive compression ratios, a fraction of prompts exhibit \emph{catastrophic failures}: the model enters infinite loops, fails to terminate, or silently corrupts its chain-of-thought reasoning.
A compressor reporting 95\% average accuracy may achieve 0\% accuracy on the 5\% of cases where compression causes catastrophe---a tail that standard evaluation completely hides.
We characterize three distinct failure profiles across three compressors: StreamingLLM exhibits a cliff (stable accuracy that collapses at a threshold), SnapKV shows catastrophic cliffs (sudden universal looping), and ObservedAttention degrades progressively.
These diverse profiles make the problem particularly challenging: a safety mechanism must handle both sudden and gradual failure modes.

All existing KV-cache compression methods are \emph{open-loop}: they decide which tokens to evict based on importance scores or attention patterns, execute the eviction irreversibly, and never check whether the decision was correct.
Recent work has begun to recognize this gap.
ASR-KF-EGR~\cite{asr_kf_egr2024} proposes but does not implement an entropy-guided recovery system.
RefreshKV~\cite{refreshkv2024} maintains the full cache for periodic recalibration, sacrificing memory savings entirely.
ERGO~\cite{ergo2025} demonstrates that entropy signals can trigger corrective action but employs only a full context reset---a sledgehammer response to what may be a localized problem.
ThinKV~\cite{thinkv2025} classifies reasoning phases to vary compression aggressiveness, but provides no mechanism to detect or recover from misclassification.
None of these systems close the loop: none monitor compressed generation quality and feed that information back into cache management.

The detection signals, however, already exist.
The HALT framework~\cite{halt2024} shows that a model's top-$k$ logit features cheaply predict generation quality.
Work on reasoning dynamics~\cite{reasoning_dynamics2025} identifies a small set of ``thinking tokens'' whose suppression disproportionately degrades chain-of-thought.
Entropy monitoring, used by ERGO for post-hoc detection, can serve as a proactive leading indicator.
Repetition patterns directly signal looping.
What is missing is not signal---it is \emph{control}: connecting these signals to a feedback mechanism that adjusts compression before failures become irreversible.

We present \textsc{kvguard}, a catastrophe-aware controller inspired by TCP congestion control~\cite{tcp_reno}.
Like TCP, the controller starts aggressively (high compression), monitors for trouble (hazard prediction), and backs off when problems are detected (graduated response from alert through safe mode to recovery).
Unlike existing systems, \textsc{kvguard} is a \emph{wrapper}: it wraps any existing KV-cache compressor without modification, making it orthogonal to compression research---every advance in compression methods automatically benefits from the safety layer.

Our contributions are:
\begin{enumerate}
    \item \textbf{Characterization of compression-induced catastrophic failures.}
    We collect 800 generation traces across three compressors and five compression ratios, identifying three distinct failure profiles and showing that 44\% of traces at moderate-to-aggressive compression exhibit catastrophe.

    \item \textbf{A zero-cost hazard predictor.}
    We extract a 42-dimensional per-token feature vector from quantities already computed during decoding (logit distribution, entropy, repetition counts) and train an XGBoost classifier that achieves AUROC 0.945 with 100\% trace-level detection and 34-token mean lead time before catastrophe onset.
    Leave-one-compressor-out cross-validation (mean AUROC 0.898) validates cross-compressor generalization.

    \item \textbf{A closed-loop controller with hysteresis-based state machine.}
    The controller reduces CFR by 43.2\% overall (63.9\% excluding immediate-onset cases) with only 6.9\% false positive rate, exceeding 50\% CFR reduction on all three compressors.
    The key mechanism is cumulative detection: requiring $k$ consecutive high-risk tokens converts modest per-token recall into reliable per-trace triggering.

    \item \textbf{Systematic ablation validating each component.}
    We show that the trained predictor adds 42.6 percentage points over a random baseline, hysteresis reduces false positives by 53pp at a 17pp prevention cost, and the balanced configuration achieves 70.5\% of the theoretical maximum prevention ceiling while triggering on only 8.1\% of prompts.
\end{enumerate}
 from main paper
% Requires: No special packages

\begin{abstract}
KV-cache compression enables long-context inference under memory constraints, but existing compressors operate without feedback: they evict tokens irreversibly and never verify whether eviction degraded generation quality.
We show that compression induces \emph{catastrophic failures}---looping, non-termination, and reasoning corruption---that standard average-accuracy metrics hide.
We present \textsc{kvguard}, a closed-loop controller inspired by TCP congestion control that wraps any existing KV-cache compressor.
\textsc{kvguard} extracts a 42-dimensional per-token feature vector from the model's own logit distribution at near-zero cost, trains a hazard predictor to forecast catastrophe within the next 32 tokens (AUROC 0.945), and uses a hysteresis-based state machine to dynamically relax compression when danger is detected.
Evaluated on Qwen2.5-3B-Instruct with GSM8K across three compressors (StreamingLLM, SnapKV, ObservedAttention) at five compression ratios, \textsc{kvguard} reduces the Catastrophic Failure Rate by 43.2\% overall (63.9\% excluding immediate-onset cases) with only 6.9\% false positive rate.
The controller exceeds 50\% CFR reduction on all three compressors, validating its compressor-agnostic design.
Ablations confirm that each component---trained predictor (+42.6pp over random), hysteresis ($-$53pp false positives), and graduated response---contributes independently.
\end{abstract}


\section{Introduction}
\label{sec:introduction}

Autoregressive language models cache key-value (KV) pairs from every attention layer at every decoding step, producing memory requirements that grow linearly with sequence length.
For long-context inference---multi-step reasoning, document summarization, agentic workflows---this cache can dominate GPU memory, prompting a line of work on KV-cache compression: methods that evict, quantize, or merge cache entries to reduce the memory footprint~\cite{streamingllm2024,snapkv2024,h2o2023,pyramidkv2024}.
These methods report average accuracy or perplexity under compression and show that significant memory savings are possible with modest average-case degradation.

However, average metrics conceal a dangerous failure pattern.
At moderate-to-aggressive compression ratios, a fraction of prompts exhibit \emph{catastrophic failures}: the model enters infinite loops, fails to terminate, or silently corrupts its chain-of-thought reasoning.
A compressor reporting 95\% average accuracy may achieve 0\% accuracy on the 5\% of cases where compression causes catastrophe---a tail that standard evaluation completely hides.
We characterize three distinct failure profiles across three compressors: StreamingLLM exhibits a cliff (stable accuracy that collapses at a threshold), SnapKV shows catastrophic cliffs (sudden universal looping), and ObservedAttention degrades progressively.
These diverse profiles make the problem particularly challenging: a safety mechanism must handle both sudden and gradual failure modes.

All existing KV-cache compression methods are \emph{open-loop}: they decide which tokens to evict based on importance scores or attention patterns, execute the eviction irreversibly, and never check whether the decision was correct.
Recent work has begun to recognize this gap.
ASR-KF-EGR~\cite{asr_kf_egr2024} proposes but does not implement an entropy-guided recovery system.
RefreshKV~\cite{refreshkv2024} maintains the full cache for periodic recalibration, sacrificing memory savings entirely.
ERGO~\cite{ergo2025} demonstrates that entropy signals can trigger corrective action but employs only a full context reset---a sledgehammer response to what may be a localized problem.
ThinKV~\cite{thinkv2025} classifies reasoning phases to vary compression aggressiveness, but provides no mechanism to detect or recover from misclassification.
None of these systems close the loop: none monitor compressed generation quality and feed that information back into cache management.

The detection signals, however, already exist.
The HALT framework~\cite{halt2024} shows that a model's top-$k$ logit features cheaply predict generation quality.
Work on reasoning dynamics~\cite{reasoning_dynamics2025} identifies a small set of ``thinking tokens'' whose suppression disproportionately degrades chain-of-thought.
Entropy monitoring, used by ERGO for post-hoc detection, can serve as a proactive leading indicator.
Repetition patterns directly signal looping.
What is missing is not signal---it is \emph{control}: connecting these signals to a feedback mechanism that adjusts compression before failures become irreversible.

We present \textsc{kvguard}, a catastrophe-aware controller inspired by TCP congestion control~\cite{tcp_reno}.
Like TCP, the controller starts aggressively (high compression), monitors for trouble (hazard prediction), and backs off when problems are detected (graduated response from alert through safe mode to recovery).
Unlike existing systems, \textsc{kvguard} is a \emph{wrapper}: it wraps any existing KV-cache compressor without modification, making it orthogonal to compression research---every advance in compression methods automatically benefits from the safety layer.

Our contributions are:
\begin{enumerate}
    \item \textbf{Characterization of compression-induced catastrophic failures.}
    We collect 800 generation traces across three compressors and five compression ratios, identifying three distinct failure profiles and showing that 44\% of traces at moderate-to-aggressive compression exhibit catastrophe.

    \item \textbf{A zero-cost hazard predictor.}
    We extract a 42-dimensional per-token feature vector from quantities already computed during decoding (logit distribution, entropy, repetition counts) and train an XGBoost classifier that achieves AUROC 0.945 with 100\% trace-level detection and 34-token mean lead time before catastrophe onset.
    Leave-one-compressor-out cross-validation (mean AUROC 0.898) validates cross-compressor generalization.

    \item \textbf{A closed-loop controller with hysteresis-based state machine.}
    The controller reduces CFR by 43.2\% overall (63.9\% excluding immediate-onset cases) with only 6.9\% false positive rate, exceeding 50\% CFR reduction on all three compressors.
    The key mechanism is cumulative detection: requiring $k$ consecutive high-risk tokens converts modest per-token recall into reliable per-trace triggering.

    \item \textbf{Systematic ablation validating each component.}
    We show that the trained predictor adds 42.6 percentage points over a random baseline, hysteresis reduces false positives by 53pp at a 17pp prevention cost, and the balanced configuration achieves 70.5\% of the theoretical maximum prevention ceiling while triggering on only 8.1\% of prompts.
\end{enumerate}
 from main paper
% Requires: No special packages

\begin{abstract}
KV-cache compression enables long-context inference under memory constraints, but existing compressors operate without feedback: they evict tokens irreversibly and never verify whether eviction degraded generation quality.
We show that compression induces \emph{catastrophic failures}---looping, non-termination, and reasoning corruption---that standard average-accuracy metrics hide.
We present \textsc{kvguard}, a closed-loop controller inspired by TCP congestion control that wraps any existing KV-cache compressor.
\textsc{kvguard} extracts a 42-dimensional per-token feature vector from the model's own logit distribution at near-zero cost, trains a hazard predictor to forecast catastrophe within the next 32 tokens (AUROC 0.945), and uses a hysteresis-based state machine to dynamically relax compression when danger is detected.
Evaluated on Qwen2.5-3B-Instruct with GSM8K across three compressors (StreamingLLM, SnapKV, ObservedAttention) at five compression ratios, \textsc{kvguard} reduces the Catastrophic Failure Rate by 43.2\% overall (63.9\% excluding immediate-onset cases) with only 6.9\% false positive rate.
The controller exceeds 50\% CFR reduction on all three compressors, validating its compressor-agnostic design.
Ablations confirm that each component---trained predictor (+42.6pp over random), hysteresis ($-$53pp false positives), and graduated response---contributes independently.
\end{abstract}


\section{Introduction}
\label{sec:introduction}

Autoregressive language models cache key-value (KV) pairs from every attention layer at every decoding step, producing memory requirements that grow linearly with sequence length.
For long-context inference---multi-step reasoning, document summarization, agentic workflows---this cache can dominate GPU memory, prompting a line of work on KV-cache compression: methods that evict, quantize, or merge cache entries to reduce the memory footprint~\cite{streamingllm2024,snapkv2024,h2o2023,pyramidkv2024}.
These methods report average accuracy or perplexity under compression and show that significant memory savings are possible with modest average-case degradation.

However, average metrics conceal a dangerous failure pattern.
At moderate-to-aggressive compression ratios, a fraction of prompts exhibit \emph{catastrophic failures}: the model enters infinite loops, fails to terminate, or silently corrupts its chain-of-thought reasoning.
A compressor reporting 95\% average accuracy may achieve 0\% accuracy on the 5\% of cases where compression causes catastrophe---a tail that standard evaluation completely hides.
We characterize three distinct failure profiles across three compressors: StreamingLLM exhibits a cliff (stable accuracy that collapses at a threshold), SnapKV shows catastrophic cliffs (sudden universal looping), and ObservedAttention degrades progressively.
These diverse profiles make the problem particularly challenging: a safety mechanism must handle both sudden and gradual failure modes.

All existing KV-cache compression methods are \emph{open-loop}: they decide which tokens to evict based on importance scores or attention patterns, execute the eviction irreversibly, and never check whether the decision was correct.
Recent work has begun to recognize this gap.
ASR-KF-EGR~\cite{asr_kf_egr2024} proposes but does not implement an entropy-guided recovery system.
RefreshKV~\cite{refreshkv2024} maintains the full cache for periodic recalibration, sacrificing memory savings entirely.
ERGO~\cite{ergo2025} demonstrates that entropy signals can trigger corrective action but employs only a full context reset---a sledgehammer response to what may be a localized problem.
ThinKV~\cite{thinkv2025} classifies reasoning phases to vary compression aggressiveness, but provides no mechanism to detect or recover from misclassification.
None of these systems close the loop: none monitor compressed generation quality and feed that information back into cache management.

The detection signals, however, already exist.
The HALT framework~\cite{halt2024} shows that a model's top-$k$ logit features cheaply predict generation quality.
Work on reasoning dynamics~\cite{reasoning_dynamics2025} identifies a small set of ``thinking tokens'' whose suppression disproportionately degrades chain-of-thought.
Entropy monitoring, used by ERGO for post-hoc detection, can serve as a proactive leading indicator.
Repetition patterns directly signal looping.
What is missing is not signal---it is \emph{control}: connecting these signals to a feedback mechanism that adjusts compression before failures become irreversible.

We present \textsc{kvguard}, a catastrophe-aware controller inspired by TCP congestion control~\cite{tcp_reno}.
Like TCP, the controller starts aggressively (high compression), monitors for trouble (hazard prediction), and backs off when problems are detected (graduated response from alert through safe mode to recovery).
Unlike existing systems, \textsc{kvguard} is a \emph{wrapper}: it wraps any existing KV-cache compressor without modification, making it orthogonal to compression research---every advance in compression methods automatically benefits from the safety layer.

Our contributions are:
\begin{enumerate}
    \item \textbf{Characterization of compression-induced catastrophic failures.}
    We collect 800 generation traces across three compressors and five compression ratios, identifying three distinct failure profiles and showing that 44\% of traces at moderate-to-aggressive compression exhibit catastrophe.

    \item \textbf{A zero-cost hazard predictor.}
    We extract a 42-dimensional per-token feature vector from quantities already computed during decoding (logit distribution, entropy, repetition counts) and train an XGBoost classifier that achieves AUROC 0.945 with 100\% trace-level detection and 34-token mean lead time before catastrophe onset.
    Leave-one-compressor-out cross-validation (mean AUROC 0.898) validates cross-compressor generalization.

    \item \textbf{A closed-loop controller with hysteresis-based state machine.}
    The controller reduces CFR by 43.2\% overall (63.9\% excluding immediate-onset cases) with only 6.9\% false positive rate, exceeding 50\% CFR reduction on all three compressors.
    The key mechanism is cumulative detection: requiring $k$ consecutive high-risk tokens converts modest per-token recall into reliable per-trace triggering.

    \item \textbf{Systematic ablation validating each component.}
    We show that the trained predictor adds 42.6 percentage points over a random baseline, hysteresis reduces false positives by 53pp at a 17pp prevention cost, and the balanced configuration achieves 70.5\% of the theoretical maximum prevention ceiling while triggering on only 8.1\% of prompts.
\end{enumerate}
 from main paper
% Requires: No special packages

\begin{abstract}
KV-cache compression enables long-context inference under memory constraints, but existing compressors operate without feedback: they evict tokens irreversibly and never verify whether eviction degraded generation quality.
We show that compression induces \emph{catastrophic failures}---looping, non-termination, and reasoning corruption---that standard average-accuracy metrics hide.
We present \textsc{kvguard}, a closed-loop controller inspired by TCP congestion control that wraps any existing KV-cache compressor.
\textsc{kvguard} extracts a 42-dimensional per-token feature vector from the model's own logit distribution at near-zero cost, trains a hazard predictor to forecast catastrophe within the next 32 tokens (AUROC 0.945), and uses a hysteresis-based state machine to dynamically relax compression when danger is detected.
Evaluated on Qwen2.5-3B-Instruct with GSM8K across three compressors (StreamingLLM, SnapKV, ObservedAttention) at five compression ratios, \textsc{kvguard} reduces the Catastrophic Failure Rate by 43.2\% overall (63.9\% excluding immediate-onset cases) with only 6.9\% false positive rate.
The controller exceeds 50\% CFR reduction on all three compressors, validating its compressor-agnostic design.
Ablations confirm that each component---trained predictor (+42.6pp over random), hysteresis ($-$53pp false positives), and graduated response---contributes independently.
\end{abstract}


\section{Introduction}
\label{sec:introduction}

Autoregressive language models cache key-value (KV) pairs from every attention layer at every decoding step, producing memory requirements that grow linearly with sequence length.
For long-context inference---multi-step reasoning, document summarization, agentic workflows---this cache can dominate GPU memory, prompting a line of work on KV-cache compression: methods that evict, quantize, or merge cache entries to reduce the memory footprint~\cite{streamingllm2024,snapkv2024,h2o2023,pyramidkv2024}.
These methods report average accuracy or perplexity under compression and show that significant memory savings are possible with modest average-case degradation.

However, average metrics conceal a dangerous failure pattern.
At moderate-to-aggressive compression ratios, a fraction of prompts exhibit \emph{catastrophic failures}: the model enters infinite loops, fails to terminate, or silently corrupts its chain-of-thought reasoning.
A compressor reporting 95\% average accuracy may achieve 0\% accuracy on the 5\% of cases where compression causes catastrophe---a tail that standard evaluation completely hides.
We characterize three distinct failure profiles across three compressors: StreamingLLM exhibits a cliff (stable accuracy that collapses at a threshold), SnapKV shows catastrophic cliffs (sudden universal looping), and ObservedAttention degrades progressively.
These diverse profiles make the problem particularly challenging: a safety mechanism must handle both sudden and gradual failure modes.

All existing KV-cache compression methods are \emph{open-loop}: they decide which tokens to evict based on importance scores or attention patterns, execute the eviction irreversibly, and never check whether the decision was correct.
Recent work has begun to recognize this gap.
ASR-KF-EGR~\cite{asr_kf_egr2024} proposes but does not implement an entropy-guided recovery system.
RefreshKV~\cite{refreshkv2024} maintains the full cache for periodic recalibration, sacrificing memory savings entirely.
ERGO~\cite{ergo2025} demonstrates that entropy signals can trigger corrective action but employs only a full context reset---a sledgehammer response to what may be a localized problem.
ThinKV~\cite{thinkv2025} classifies reasoning phases to vary compression aggressiveness, but provides no mechanism to detect or recover from misclassification.
None of these systems close the loop: none monitor compressed generation quality and feed that information back into cache management.

The detection signals, however, already exist.
The HALT framework~\cite{halt2024} shows that a model's top-$k$ logit features cheaply predict generation quality.
Work on reasoning dynamics~\cite{reasoning_dynamics2025} identifies a small set of ``thinking tokens'' whose suppression disproportionately degrades chain-of-thought.
Entropy monitoring, used by ERGO for post-hoc detection, can serve as a proactive leading indicator.
Repetition patterns directly signal looping.
What is missing is not signal---it is \emph{control}: connecting these signals to a feedback mechanism that adjusts compression before failures become irreversible.

We present \textsc{kvguard}, a catastrophe-aware controller inspired by TCP congestion control~\cite{tcp_reno}.
Like TCP, the controller starts aggressively (high compression), monitors for trouble (hazard prediction), and backs off when problems are detected (graduated response from alert through safe mode to recovery).
Unlike existing systems, \textsc{kvguard} is a \emph{wrapper}: it wraps any existing KV-cache compressor without modification, making it orthogonal to compression research---every advance in compression methods automatically benefits from the safety layer.

Our contributions are:
\begin{enumerate}
    \item \textbf{Characterization of compression-induced catastrophic failures.}
    We collect 800 generation traces across three compressors and five compression ratios, identifying three distinct failure profiles and showing that 44\% of traces at moderate-to-aggressive compression exhibit catastrophe.

    \item \textbf{A zero-cost hazard predictor.}
    We extract a 42-dimensional per-token feature vector from quantities already computed during decoding (logit distribution, entropy, repetition counts) and train an XGBoost classifier that achieves AUROC 0.945 with 100\% trace-level detection and 34-token mean lead time before catastrophe onset.
    Leave-one-compressor-out cross-validation (mean AUROC 0.898) validates cross-compressor generalization.

    \item \textbf{A closed-loop controller with hysteresis-based state machine.}
    The controller reduces CFR by 43.2\% overall (63.9\% excluding immediate-onset cases) with only 6.9\% false positive rate, exceeding 50\% CFR reduction on all three compressors.
    The key mechanism is cumulative detection: requiring $k$ consecutive high-risk tokens converts modest per-token recall into reliable per-trace triggering.

    \item \textbf{Systematic ablation validating each component.}
    We show that the trained predictor adds 42.6 percentage points over a random baseline, hysteresis reduces false positives by 53pp at a 17pp prevention cost, and the balanced configuration achieves 70.5\% of the theoretical maximum prevention ceiling while triggering on only 8.1\% of prompts.
\end{enumerate}
